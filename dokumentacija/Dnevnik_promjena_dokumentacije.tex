\chapter{Dnevnik promjena dokumentacije}
		
		%\textbf{\textit{Kontinuirano osvježavanje}}\\
				
		
		\begin{longtblr}[
				label=none
			]{
				width = \textwidth, 
				colspec={|X[2]|X[13]|X[3]|X[3]|}, 
				rowhead = 1
			}
			\hline
			\textbf{Rev.}	& \textbf{Opis promjene/dodatka} & \textbf{Autori} & \textbf{Datum}\\[3pt] \hline
			0.1 & Napravljen predložak	& Mario Petek & 30.10.2022. \\[3pt] \hline 
			0.2	& Napisan opis projektnog zadatka & Ivan Kuzmić, Mario Petek & 31.10.2022. 	\\[3pt] \hline 
			0.3 & Napisani funkcijski zahtjevi & Ivan Kuzmić, Mario Petek & 01.11.2022. \\[3pt] \hline 
			0.4.1 & Napisan dio obrazaca uporabe & Ivan Kuzmić, Mario Petek & 01.11.2022. \\[3pt] \hline 
			0.4.2 & Ažuriranje obrazaca uporabe & Antonio Lukić, Mario Petek & 02.11.2022. \\[3pt] \hline 
			0.4.3 & Napisan ostatak obrazaca uporabe i popravljanje grešaka & Ivan Kuzmić & 05.11.2022. \\[3pt] \hline 
			0.4.4 & Dodavanje potrebnih informacija obrascima uporabe & Ivan Kuzmić & 11.11.2022. \\[3pt] \hline 
			0.5 & Dodavanje dijagrama obrazaca uporabe i sekvencijskih dijagrama & Ivan Kuzmić, Mario Petek & 14.11.2022. \\[3pt] \hline
			0.6 & Dodavanje ostalih zahtjeva, arhitekture i opisa baze podataka & Ivan Kuzmić & 14.11.2022. \\[3pt] \hline
			0.7 & Dodavanje dijagrama razreda & Antonio Lukić & 16.11.2022. \\[3pt] \hline
			\textbf{1.0} & Verzija samo s bitnim dijelovima za 1. ciklus & Ivan Kuzmić & 18.11.2022. \\[3pt] \hline 
			1.1 & Izmjena napomenutih dijelova iz prvog ocjenjivanja & Ivan Kuzmić, Mario Petek & 03.01.2022. \\[3pt] \hline 
			1.2 & Dodavanje ostatka dijagrama razreda & Antonio Lukić & 04.01.2022. \\[3pt] \hline 
			1.3 & Dodavanje dijagrama aktivnosti & Antonio Lukić & 07.01.2022. \\[3pt] \hline
			1.4 & Dodavanje prve verzije dijagrama stanja & Mario Petek & 09.01.2022. \\[3pt] \hline
			1.5 & Dodavanje korištenih tehnologija i alata & Mario Petek & 09.01.2022. \\[3pt] \hline
			\textbf{2.0} & Konačni tekst predloška dokumentacije  &  &  \\[3pt] \hline 
		\end{longtblr}
	
	
		%\textit{Moraju postojati glavne revizije dokumenata 1.0 i 2.0 na kraju prvog i drugog ciklusa. Između tih revizija mogu postojati manje revizije već prema tome kako se dokument bude nadopunjavao. Očekuje se da nakon svake značajnije promjene (dodatka, izmjene, uklanjanja dijelova teksta i popratnih grafičkih sadržaja) dokumenta se to zabilježi kao revizija. Npr., revizije unutar prvog ciklusa će imati oznake 0.1, 0.2, …, 0.9, 0.10, 0.11.. sve do konačne revizije prvog ciklusa 1.0. U drugom ciklusu se nastavlja s revizijama 1.1, 1.2, itd.}